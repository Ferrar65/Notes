\documentclass{article}

% Language setting
% Replace `english' with e.g. `spanish' to change the document language
\usepackage[english]{babel}

% Set page size and margins
% Replace `letterpaper' with`a4paper' for UK/EU standard size
\usepackage[letterpaper,top=2cm,bottom=2cm,left=3cm,right=3cm,marginparwidth=1.75cm]{geometry}

% Useful packages
\usepackage{amsmath}
\usepackage{graphicx}
\usepackage[colorlinks=true, allcolors=blue]{hyperref}

\title{Prova Parallela Ed.Civica/Storia}
\author{Karim El-Metkoul}

\begin{document}
\maketitle
\section{Le origini della costituzione dell'italia repubblicana}
\subsection{Tesi di fondo Autore:}
L’opera rappresenta l'avvenimento legislativa molto importante per 
la costituzione italiana, il progetto è quello di consentire la realizzazione di uno stato unito.
Sulla carta l’elaborazione ebbe un certo riguardo alla tutela dei diritti sociali, allo scopo di assicurare una vita “dignitosa” e piena di partecipazione alla vita (sociale ed economica e dunque) politica del Paese.
Ciò creò delle incongruenze nell'introduzione di tutte quelle norme sistematiche, causando l'infondata utilità delle stesse.
 Nel corso del processo la “costituzione-italiana” mostrava ideali e aspirazioni solo alla egemonia dunque alla parte più avanzata del popolo.
Creando un clima molto simile al suffragio censitario, dove solo chi aveva una quantità economica poteva mostrare e portare avanti un proprio pensiero, rappresentando dunque solo la media-alta società.
tale problema dell’attuazione costituzionale si pose al centro della lotta politica, per generare trasformazioni strutturali all’intera società italiana nel dopoguerra, per una crescita civile dalle fondamenta attraverso la partecipazione dei cittadini, alla formazione della volontà generale.
\section{Suddivisione Paragrafi:}
\begin{itemize}
    \item Riga 1;8 : Introduzione
    \item Riga 8;14: Il Motivo il quale scaturì le contraddizioni
    \item Riga 14;20: Nascita del progetto di costituzione
    \item Riga 20;29:  insurrezione delle prime contraddizioni costituzionali.
    \item Riga 29;40: Lo sviluppo della costituzione i primi cambiamenti nella società
\end{itemize}
\section{Sviluppo argomentativo paragrafo:}
\subsection{L'introduzione}preludia la rottura dell’unita antifascista riferendosi alla fine dei governi costituiti da tutte le forze politiche dando un notevole rilievo alla realizazione della costituzione.
\subsection{Il Motivo il quale scaturì le contraddizioni:}
Le contraddizioni  venerò indotte dalla consapevolezza dei componenti della resistenza politica che avevano radicato la società nel fascismo.
Grazie al movimento dell’antifascismo si trovò un’intesa che permise 
l’adempimento di un primo tentativo di legislazione costituzionale.
\subsection{Nascita del progetto di costituzione:}
Il progetto di costituzione si impegnò a definire e tutelare non solo i diritti di libertà, ma anche i diritti sociali per assicurare una vita “dignitosa” al Paese.
\subsection{L’Insurrezione delle prime contraddizioni costituzionali:}
L’introduzione dei “diritti sociali” non fù pacifica, questo dettato perché il futuro legislatore dovrà avere un certo riguardo al futuro obbligandolo ad avere una consapevolezza normativa programmatica alle norme vigenti. Generando critiche fondate sulle tecniche giuridiche, fornendo una maggior responsabilità nei confronti del popolo.
\subsection{ Lo sviluppo della costituzione i primi cambiamenti nella società:}
La realtà costituzionale penetrò nelle grandi masse, dando luogo all'assimilazione degli ideali.
Suscitando una vera e propria trasformazione della società italiana nel secondo dopoguerra. Avviando una sempre più forte partecipazione dei cittadini al rispetto delle leggi di comportamento responsabile ed eticamente corretto.
\section{Commento in merito alle posizioni emergenti, motivando.}
\section{Condividi l'opinione secondo cui la nuova costituzione ha segnato una rottura con lo stato liberale dell'epoca precedente al fascismo e ha condizionato la successiva storia d'italia?}
\section{Ritieni che il problema dell’attuazione della costituzione sia ancora al centro del dibattito politico?
No la costituzione italiana è stata calpestata!}





\end{document}
\documentclass{article}

% Language setting
% Replace `english' with e.g. `spanish' to change the document language
\usepackage[english]{babel}

% Set page size and margins
% Replace `letterpaper' with`a4paper' for UK/EU standard size
\usepackage[letterpaper,top=2cm,bottom=2cm,left=3cm,right=3cm,marginparwidth=1.75cm]{geometry}

% Useful packages
\usepackage{amsmath}
\usepackage{graphicx}
\usepackage[colorlinks=true, allcolors=blue]{hyperref}

\title{APPUNTI - Letteratura}
\author{Karim El-Metkoul}

\begin{document}
\maketitle
\section{Positivismo}

\begin{itemize}
    \item La religione cosa fa? Fonda tutto su un atto di fede.
    \item La società da fede incondizionata alla scienza e la tecnica.
    \item Auguste Comte, dice basta occuparsi alla filosofia, non bisogna chiedersi delle domande
    Misteriosi, indecifrabili, bisogna ripiegare la nostra intelligenza alla scienza. 
    \item La filosofia serve per generare certezze.
\end{itemize}
\vspace{.5cm}
La filosofia è molto teorica e deve convertirsi in prassi.
La MetaFisica deve limitarsi alle possibilità dell'uomo.

\underline{L'unica attenzione dell'uomo deve esser quella di migliorare le proprie situazioni nella società}

Bisogna andare nel concreto. Niente deduzioni.
La nuova filosofia dovrà essere ottimista, dove vi sarà benessere uguaglianza.
La società intende che il passato è inutile.
A cosa serve la storia? 
Comte-Darwin-Nietzsche 
Determinismo storico = se sono così è perché sono nato in certe condizioni/ambiente.
Comte = l'unica sapienza è quella scientifica.

\textbf{Ci sono due entità nella storia,  Individuo e la società.}\\
Il debole tendere sempre ad esser divorato, questa concezione della società nel futuro si applicherà anche nella società con distinzioni razziali sopratutto nei totalitarismi.
\section{Divisione in Razze}
\textbf{Il razzismo non potrà mai essere un sentimento}
Solo nel 800 iniziano le prime classificazioni in base alla razza, grazie al Positivismo.
La germania inizio ad identificarsi come un popolo sacro, a definirsi superiori rispetto ai altri popoli.
\begin{itemize}
    \item Wagner parlò di un popolo superiore rispetto ai altri.
    \item Verdi invece parlo di come vivevano gli ebrei.
    \item Verdi odio le opere di Wagner perché rappresento e diffuse l'dio e lo stigma Xenofobico.
\end{itemize}
Il successo del razzismo venne causato da una isteria del popolo.
\begin{itemize}
    \item Studia Pag. 59|60 / Appunti
\end{itemize}
{\Large\textbf{Schemappa:}}
\begin{itemize}
\item Schemappa:
\item Positivismo
\item DarwinismoSociale
\item Chamberlains
\item Razzismo scientifico / sociale
\end{itemize}
Suffragio Censitario, Solo chi aveva una quantità economica potevano votare, Solo la Media-Alta società poteva
votare e presentare la propria idea. 'La classe era ristrettissime'
\section{La crisi di fine secolo 1890}
Ci furono movimenti di opposizione contro la classe liberale e dei borghesi, Volevo solo il diritto al Lavoro.\\
29 Luglio 1900 a Monza il re Umberto primo 

\section{Scapiliatura}
Corrente letteraria artistica 1860 - 1880 italia\\
Cosa succede tra il 1860-1880 in italia?\\
17 Marzo 1861 - Proclamazione del regno d'italia \textrightarrow Vittorio Emanuele II Re d'italia
\textbf{Destra storica al potere fino al 1876}
\textbf{Unificazione Italia}
\begin{itemize}
    \item Roma e Lazio \textrightarrow Presa al potere 20 Set 1870
    \item Trentino A | Fiuli V.G
    \item Veneto \textrightarrow 1866 ('Terza guerra d'indipendenza')
    \item Questione Meridionale
    \item \textbf{Debito Economico Costruzione Obbligatoria - tassa sul macinato}
\end{itemize}





\end{document}
\documentclass{article}

% Language setting
% Replace `english' with e.g. `spanish' to change the document language
\usepackage[english]{babel}

% Set page size and margins
% Replace `letterpaper' with`a4paper' for UK/EU standard size
\usepackage[letterpaper,top=2cm,bottom=2cm,left=3cm,right=3cm,marginparwidth=1.75cm]{geometry}

% Useful packages
\usepackage{amsmath}
\usepackage{graphicx}
\usepackage[colorlinks=true, allcolors=blue]{hyperref}

\title{APPUNTI - ITALIANO/STORIA}
\author{Karim El-Metkoul}

\begin{document}
\maketitle

\section{Nuovo Testo-Comprensione}
Dopo aver analizzato il brano, un predatore nel'acquario, dopo aver ragionato sui contenuti
 presenti nello stesso, prova a riflettere sulla situazione della società odierna, (Micro-Macro società).
Può esser considerata simbolicamente un'acquario dove vivono terribili predatori?
\\Esistono dei Dytiscus spietati e voraci?
elabora la tua tesi a partire da un introduzione, di carattere storico (le differenze tra le quarto rivoluzioni industriali; La nascita dei partiti di massa, la presa di conoscenza dei propri diritti , la dottrina sociale della chiesa, la società di massa.)

Propria opinione attraverso testo argomentativo, Segui il seguente schema obbligatorio.\\

\graphicspath{ {image.png} }

\begin{itemize}
    \item Introduzione di carattere espositivo, informativo, storico.
    \item Definizione e approfondimento del problema enunciato nella traccia.
    \item Domanda retorica che introduce la propria tesi. 
    \item Enunciazione della propria tesi chiara sottolineata.
    \item Argomentazione a sostegno della tesi ("La tesi, deve essere avvalorata da citazione di personaggi competenti e dalla propria esperienza personale, ricorda sempre di riportare in calce 
    in pie di pagina la fonte")
    \item Conclusione o possibile soluzione del problema.

\end{itemize}   

\begin{enumerate}
  \item bisogna riprendere ciò che abbiamo fatto con la società di massa
  \item bisogna introdurle il problema nella prima parte "cosa centra un predatore nell'acquario e cosa centra con la società di massa,
  la piccola società e la grande società, può esser considerata un grande acquario con un predatore,
  predatore che inject veleno e succhia dai pesci" 
\end{enumerate}



\vspace{2cm}

\subsection{Desideri delle persone}

Il tempo libero manca sempre di più impegnati nel aumentare la disponibilità liquida
nelle tasche da poterla utilizzare girovagando per il mondo e divertendoci senza badare
a spese. I salari sono sempre più bassi mentre i desideri hanno forme più ampie e costose.
Spesso si sente parlare di sacrifici che vengono fatti fatti dai nostri genitori o
addirittura da noi stessi per poterci permettere una vacanza con i nostri amici come
regalo per la fine della scuola o più semplicemente un piccolo svago che si pensa di
meritare. 
Non tutti sanno cosa è il sacrificio, infatti, molti ragazzi hanno già tutto pronta
e non vogliono fare nulla da soli. Uno dei sistemi usato dal  per attrarre La nostra
 attenzione è la pubblicità. Proviamo ad accendere la televisione e proviamo ad osservare
 quello che ci appare davanti: l'immagine. Sicuramente l'invenzione della comunicazione
 visiva attraverso la televisione ha segnato una tappa fondamentale per il progresso e
 la tecnologia, eppure il mondo dell'immagine nasconde limiti e insidie di cui l'uomo
 spesso non si rende conto. La pubblicità, quindi ci induce sempre a spendere denaro,
 però bisogna precisare che questo lusso non è permesso a tutti ma solo alla popolazione
 mi media rendita. Ci nono molte famiglie che pur lavorando fanno fatica perfino ad 
 acquistare le cose primarie come materiale scolastico per i figli o vestiti invernali 
 caldi. Le varie aziende sono spietate con i prezzi soprattutto in determinati periodi
 dell’anno come natale, pasqua o Black Friday dove i prezzi dovrebbero essere dimezzati
 e non triplicati! Alcuni desideri sono talmente forti che a volte non si ragiona come 
 lo si dovrebbe fare e si rischia di indebitarsi talmente tanto da perdere tutto quello
 che si ha persino anche i familiari. La nostra mente ormai è annebbiata dal desiderio,
 il consumismo scorre dentro di noi e alla prima cosa bella ci soffermiamo e la pretendiamo
  come se fosse l’unica cosa che ci manterrà in vita. Il mondo è troppo viziato per capire
  con chiarezza di cosa ha realmente bisogno.

\vspace{1cm}
Da completare ed evidenzia la tua tesi.

La società di massa, vengono concepite 
\subsection{La società di massa, nella prima rivoluzione industriale}

con la seconda rivoluzione industriale e l'avvento della società di massa, 
si cercò di razionalizzare il lavoro nelle fabbriche: il fenomeno più importante
fu l'introduzione, nel 1913, della catena di montaggio, nella fabbrica di automobili
Ford, poi imitata da moltissime industrie. Con la catena di montaggio ogni operaio
non doveva più lavorare alla produzione dell'intero prodotto, ma compiere solo una
piccola parte del lavoro necessario a tale fine, in modo ripetitivo e spersonalizzato.
Tale sistema assicurava una migliore resa del lavoro degli operai, ma rendeva la loro
attività lavorativa meno gratificante, più meccanica.
Sempre nella stessa direzione di razionalizzazione del lavoro venne poi elaborato il metodo
di Taylor o taylorismo, che disciplinava rigorosamente i tempi del lavoro degli operai, in
modo da evitare sprechi di tempo, come pause ingiustificate.
Non si trattò tuttavia solo di un peggioramento della condizione lavorativa dei salariati,
possiamo ricordare ad esempio che il fordismo, dal nome dell'industria di Detroit già
ricordata, prevedeva prezzi competitivi dei propri prodotti e salari alti dei propri
operai, in modo che questi potessero diventare dei consumatori.
 
\section{Testo-Comprensione}
\begin{enumerate}
  \item uso del dizionario, imparare a conoscere il significato delle parole non conosciute.
  \item divisione in sequenze e titolazione delle stesse, (“UN UN TITOLO MA UN RIASSUNTO DELLE STESSE”).
  \item sintesi del brano, massimo 12 righe
\end{enumerate}
\vspace{2cm}
\begin{itemize}

\item \textbf{Dispiega:} il termine indica l’apertura in tutta la lunghezza o la larghezza ,
Il termine può anche indicare in modo meno convenzionale una diffusione di qualcosa.
\item \textbf{Eterogeneo:} il termine indica la natura, la qualità di un oggetto o sistema nel esser 
Costituito da elementi diversi e non armonizzati tra di loro.
\item \textbf{Larva:} è uno dei stadi del primo sviluppo post-embrionale degli animali ovipari, dunque soggetti a metamorfosi.
\item \textbf{Dytiscus:} è un insetto presente nella famiglia dei coleotteri acquatici, la sua lunghezza
Può variare da circa 3.5 cm  fino a 6
La forma del Dytiscus è ovale allungata, bruna e scura caratterizzata da riflessi verdastri
E strisce gialle.
\item \textbf{Eclissando:}  il termine indica l’offuscare della luce a causa di un oggetto/entità. Esempio 
La luna che eclissa la luce del sole.
Il termine può anche indicare il nascondere di un entità.
\item \textbf{Pinne setolose:} le pinne sono appendici del corpo dei pesci che provvedono al movimento 
Contribuendo anche alla funzione stabilizzatrice dell’animale nell’acqua.
Le pinne setolose, sono pinne con segmenti/appendici dispiegate come delle spazzole.
\item \textbf{L’ingestione:} Indica l’atto di ingerire, L’ingestione non indica solo l’atto di ingerire ma indica sopratutto il fatto fisiologico riguardanti agli effetti prodotti di ciò che si è ingerito,.
\item \textbf{Secrezione Ghiandolare:} Le ghiandole sono organi secretori che servonño alla liberazione di utili all’organismo, liberando sostanza utili per le diverse attività dell’organismo.
\item \textbf{Opaco:} L'opacità di un corpo indica la proprietà di esso dic riflettere o meno le radiazioni specialmente quelle luminose. 
\item \textbf{Flaccido involucro:} indica un rivestimento esterno per lo più con funzioni protettive con una struttura scarsamente consistente, molle o cascante.

\subsection{}


autoritarismo / Autorevolezza

Autoritarismo = parte dall'alto è investe il basso. eq.(Professore che impone di fare un test, zittendo lo studente.)

Autorevolezza = parte dal basso, opera nella giustizia. si riconosce l'autorità del basso verso il basso

Gioliti = cerca l'appoggio dei socialisti, ma i socialisti non entrano nel governo
l'eta giolitiana: 

\textbf{L'italia era divisa in due, Nord e Sud:}
\begin{itemize}
  \item Nord evoluto, nei servizi e nell'attività agrarie
  \item Sud, Arretrato.
\end{itemize}




\end{document}
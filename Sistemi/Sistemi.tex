\documentclass{article}

% Language setting
% Replace `english' with e.g. `spanish' to change the document language
\usepackage[english]{babel}

% Set page size and margins
% Replace `letterpaper' with`a4paper' for UK/EU standard size
\usepackage[letterpaper,top=2cm,bottom=2cm,left=3cm,right=3cm,marginparwidth=1.75cm]{geometry}

% Useful packages
\usepackage{amsmath}
\usepackage{graphicx}
\usepackage[colorlinks=true, allcolors=blue]{hyperref}

\title{APPUNI - SISTEMI}
\author{Karim El-Metkoul  }

\begin{document}
\maketitle

\section{TCP - UDP}
TCP: Orientato alla connessione UDP: Non richiede una connessione
Protocollo TFTP su rete locale non poggia su TCP ma su UDP perch`e la
rete locale ha una bassa latenza e una bassa probabilit`a di errori.
\\

Anche il protocollo DNS su rete locale usa UDP, mentre su internet usa
TCP. Due server DNS tra loro non dialogano un UDP ma in TCP, sempre sulla
porta 53.
\\
Il DNS `e un protocollo critico perch`e i DNS di tutto il mondo devono parlarsi
e scambiarsi informazioni. \\I regimi dittatoriali configurano i DNS presenti sul
territorio in modo che non diano tutte le informazioni.
Ognuno si deve fidare delle informazioni che gli vengono rilasciate dall'altro.
Ci sono delle autorità superiori che controllano chi possiede i DNS. Se si verificano delle anomalie, le autorità riescono a risalire a chi ha creato le anomalie.
Socket = IP + Porta L’header del TCP normalmente ha 20 bytes + da 0 a
32 bytes opzionali.\\
Sequence number = TCP tiene sotto controllo il flusso e da un ordine ai
pacchetti. Acknowledgement number = chi riceve i dati comunica a chi li ha
trasmessi il corretto recapito dei dati.\\
Bit di controllo = quali funzioni sono attive in quel pacchetto. Ad esempio,
se il pacchetto `e parte di un segmento fragmentato oppure no. Oppure se il
pacchetto contiene oppure no un ACK number.\\
Window = delimitare il numero massimo di pacchetti che possono essere
spediti senza ricevere un ACK.\\
Checksum = controllo degli errori nella trasmissione (sia header che dati)
Selective ACK = vengono rinviati solo i segmenti persi e non tutti, come `e
di default.\\
Durante la fase di connessione (3 way handshake) si settano anche alcuni
parametri della comunicazione che possono essere resettati durante la connessione. Ad esempio windows size dove ci si mette d’accordo sul numero massimo
di segmenti che possono essere spediti prima di ricevere l’ACK.
Se in un determinato momento ci sono tanti host che parlano con un server,
il suo buffer si riempie e il server manda dei segmenti chiedendo di ridurre la
finestra dei pacchetti massimi che si possono spedire ogni ACK.\\
1
I protocolli di oggi utilizzano le sliding windows e inviano un ACK ogni due
segmenti che ricevono.\\
Le dimensioni della finestra non vengono definite in termini di segmenti ma
in termini di bytes, quindi la finestra viene modificata ogni due segmenti.
Tipicamente la dimensione massima di un segmento `e di 1460 bytes ma pu`o
essere modificato nel corso della connessione.
Alla fine dipende sempre dai tempi.
Guardare a casa il capitolo 15.\\

\section{Dominio di coalizione}

\subsection{Cyber-Security Cisco}

Un Tecnico deve conoscere gli strumenti necessari per proteggere i propri individui.
Le tecnologie nel tempo cambiano, perché le tecnologie continuano a cambiare nel tempo con le scoperte di nuove falle.
\\È inutile sapere un metodo per attaccare a livello pratico ma bisogna continuare ad aggiornarsi.
\\l'attacco di una rete possono dividersi in " White-Hats / Black-Hats " 
\\I White-Hats controllano vulnerabilità per poi riferirle alle aziende.

\subsection{Fase attacco di rete}

la prima fase è quella di ricognizione, Bisogna prendere informazioni. 
Bisogna informarsi sul Software-Hardware della azienda, 

Configurazioni-accessi-servizi-applicativi
\\Quando si conoscono le informazioni di un OS, si possono fare ricerche per trovare tool per aggirare il sistema operativo
Molti attacchi provengono dal personale interno alla rete, Bisogna mettere un blocco per impossibilitare il personale ad aprire documentazione

L'amministratore di rete: Deve tenere sempre sottocontollo lo stato dei device della sua rete.
\\\\
SNMP = ('Simple Network Management Protocol');
Però questo servizio presenta falle, dunque bisogna bilanciare le risorse e la sicurezza di esse.

\section{Capitolo 17}

Disaster Recovery = aggiungere ridondanza alle connessioni.
Sicurezza ridurre servizi in rete.

CDP = protocollo che permettete di verificare la salute dei dispositivi di rete connessi al router.

in molte aziende non hanno la presenza di device avendo tutto in cloud, le concessioni telefoniche (voip) e ip-phone.






\end{document}